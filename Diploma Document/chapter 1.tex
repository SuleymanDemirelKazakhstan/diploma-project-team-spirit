\chapter{Project details}\label{ch:A}
\section{Research}
Before starting the project, we consulted with many people. Since it isn't a simple process to create a web application. After that, we came to the decision to do detailed analysis of our next steps of the project. For that, we divided into several steps:
    
    1. What is our project and its main purpose and what target audience will use the application
    
    2. What features will there be 

    3. What will we have to do to achieve our goals

    4. What technologies will be used

Through the research, we found out the approximate number of people who are interested in this area and approximately how many people will use our application. Because of this, we began to understand how to move on. We took note of what problems might be in the future and began to understand how to implement our project.

The next phase of the project is market analysis. Where each of us should explore other alternatives and describe exactly how their system works, the pros and cons, as well as the errors of the services.

Further, due to the analysis, we decided to make a web-site. Where access will be for all users, regardless of the operating system of the device.

And finally, our goal is to make a good product that will be useful for people. As we said earlier, second-hand shops are very useful for the environment and we hope that our project will be useful for people and it will become more convenient for them to purchase things. Our team aims to develop this project further and will add new features in future.

%Let us cite some resources form the  bibliography. This is a a good book \cite{dirac}. Let's cite Einstein's paper \cite{einstein} and Knuth's website \cite{knuthwebsite}. Knuth has also a book on algorithms \cite{knuth-fa}.%
\section{Pain points and importance of the project}
Initially, when our team was on process of discussion of the project and the main theme. The first pain point was auction-system.

Since there is no such platform in Kazakhstan, it was not entirely clear how to implement the auction and store system at the same time. 
The solution to this problem was foreign alternative websites. More specifically, eBay.
Where the seller enters the initial amount of money and sets the period for which this lot can be bid.

The second pain point. Our problem was to find people who had experience with vintage items, as the second hand in Kazakhstan are not so popular. But we solved this problem in the following way, we started writing to people who were subscribed to the account of vintage stores. Many of them agreed to take the survey, which helped to understand the market situation. And also communicate with some buyers, and found out that the demand for vintage items increased from year to year. Thanks to the survey, we found out that most people are not comfortable looking for things in a standard way, as it takes them a lot of time and effort, but this also applies to store owners. Our main goal is to improve this process by creating a platform that automates this process without changing the fundamental business logic.

\section{Planning and defining risks}
Since every project has its own problems, we decide what method our team should choose, so that our team can quickly sort out the problems. 

We needed a method which would help our team work together. Our team was thinking about the most convenient method and also not too hard to work with, since we are just starting our journey in the IT sphere.  

Before starting the project, we thought for a long time about which management method is most suitable. Our choice fell into the scrum method.

There are many methods like Agile, Kanban, etc. But for us the most effective was scrum. 
Scrum is a setting of meeting, planning and tools in a team, that helps to better manage the work in teamwork. 
Scrum - allows us to adapt to constantly changing conditions of the project.
This method has its own artifacts or also called tools.  
Scrum have 3 solving tools:

-Backlog of project

-Backlog of sprint

-Increment. 

As for the juniors, it was the best method for us. 
Our team had everyday short calls, where every member of the team discussed their progression. And if someone had problems or some misunderstanding, they 
discussed it with the project manager. 

We look forward to implementing more new features in our platform and hope that our project will continue to grow. 



